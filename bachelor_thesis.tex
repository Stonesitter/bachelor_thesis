% Options for packages loaded elsewhere
\PassOptionsToPackage{unicode}{hyperref}
\PassOptionsToPackage{hyphens}{url}
%
\documentclass[
  man]{apa7}
\usepackage{amsmath,amssymb}
\usepackage{iftex}
\ifPDFTeX
  \usepackage[T1]{fontenc}
  \usepackage[utf8]{inputenc}
  \usepackage{textcomp} % provide euro and other symbols
\else % if luatex or xetex
  \usepackage{unicode-math} % this also loads fontspec
  \defaultfontfeatures{Scale=MatchLowercase}
  \defaultfontfeatures[\rmfamily]{Ligatures=TeX,Scale=1}
\fi
\usepackage{lmodern}
\ifPDFTeX\else
  % xetex/luatex font selection
\fi
% Use upquote if available, for straight quotes in verbatim environments
\IfFileExists{upquote.sty}{\usepackage{upquote}}{}
\IfFileExists{microtype.sty}{% use microtype if available
  \usepackage[]{microtype}
  \UseMicrotypeSet[protrusion]{basicmath} % disable protrusion for tt fonts
}{}
\makeatletter
\@ifundefined{KOMAClassName}{% if non-KOMA class
  \IfFileExists{parskip.sty}{%
    \usepackage{parskip}
  }{% else
    \setlength{\parindent}{0pt}
    \setlength{\parskip}{6pt plus 2pt minus 1pt}}
}{% if KOMA class
  \KOMAoptions{parskip=half}}
\makeatother
\usepackage{xcolor}
\usepackage{graphicx}
\makeatletter
\def\maxwidth{\ifdim\Gin@nat@width>\linewidth\linewidth\else\Gin@nat@width\fi}
\def\maxheight{\ifdim\Gin@nat@height>\textheight\textheight\else\Gin@nat@height\fi}
\makeatother
% Scale images if necessary, so that they will not overflow the page
% margins by default, and it is still possible to overwrite the defaults
% using explicit options in \includegraphics[width, height, ...]{}
\setkeys{Gin}{width=\maxwidth,height=\maxheight,keepaspectratio}
% Set default figure placement to htbp
\makeatletter
\def\fps@figure{htbp}
\makeatother
\setlength{\emergencystretch}{3em} % prevent overfull lines
\providecommand{\tightlist}{%
  \setlength{\itemsep}{0pt}\setlength{\parskip}{0pt}}
\setcounter{secnumdepth}{-\maxdimen} % remove section numbering
% Make \paragraph and \subparagraph free-standing
\ifx\paragraph\undefined\else
  \let\oldparagraph\paragraph
  \renewcommand{\paragraph}[1]{\oldparagraph{#1}\mbox{}}
\fi
\ifx\subparagraph\undefined\else
  \let\oldsubparagraph\subparagraph
  \renewcommand{\subparagraph}[1]{\oldsubparagraph{#1}\mbox{}}
\fi
% definitions for citeproc citations
\NewDocumentCommand\citeproctext{}{}
\NewDocumentCommand\citeproc{mm}{%
  \begingroup\def\citeproctext{#2}\cite{#1}\endgroup}
\makeatletter
 % allow citations to break across lines
 \let\@cite@ofmt\@firstofone
 % avoid brackets around text for \cite:
 \def\@biblabel#1{}
 \def\@cite#1#2{{#1\if@tempswa , #2\fi}}
\makeatother
\newlength{\cslhangindent}
\setlength{\cslhangindent}{1.5em}
\newlength{\csllabelwidth}
\setlength{\csllabelwidth}{3em}
\newenvironment{CSLReferences}[2] % #1 hanging-indent, #2 entry-spacing
 {\begin{list}{}{%
  \setlength{\itemindent}{0pt}
  \setlength{\leftmargin}{0pt}
  \setlength{\parsep}{0pt}
  % turn on hanging indent if param 1 is 1
  \ifodd #1
   \setlength{\leftmargin}{\cslhangindent}
   \setlength{\itemindent}{-1\cslhangindent}
  \fi
  % set entry spacing
  \setlength{\itemsep}{#2\baselineskip}}}
 {\end{list}}
\usepackage{calc}
\newcommand{\CSLBlock}[1]{\hfill\break\parbox[t]{\linewidth}{\strut\ignorespaces#1\strut}}
\newcommand{\CSLLeftMargin}[1]{\parbox[t]{\csllabelwidth}{\strut#1\strut}}
\newcommand{\CSLRightInline}[1]{\parbox[t]{\linewidth - \csllabelwidth}{\strut#1\strut}}
\newcommand{\CSLIndent}[1]{\hspace{\cslhangindent}#1}
\ifLuaTeX
\usepackage[bidi=basic]{babel}
\else
\usepackage[bidi=default]{babel}
\fi
\babelprovide[main,import]{english}
% get rid of language-specific shorthands (see #6817):
\let\LanguageShortHands\languageshorthands
\def\languageshorthands#1{}
% Manuscript styling
\usepackage{upgreek}
\captionsetup{font=singlespacing,justification=justified}

% Table formatting
\usepackage{longtable}
\usepackage{lscape}
% \usepackage[counterclockwise]{rotating}   % Landscape page setup for large tables
\usepackage{multirow}		% Table styling
\usepackage{tabularx}		% Control Column width
\usepackage[flushleft]{threeparttable}	% Allows for three part tables with a specified notes section
\usepackage{threeparttablex}            % Lets threeparttable work with longtable

% Create new environments so endfloat can handle them
% \newenvironment{ltable}
%   {\begin{landscape}\centering\begin{threeparttable}}
%   {\end{threeparttable}\end{landscape}}
\newenvironment{lltable}{\begin{landscape}\centering\begin{ThreePartTable}}{\end{ThreePartTable}\end{landscape}}

% Enables adjusting longtable caption width to table width
% Solution found at http://golatex.de/longtable-mit-caption-so-breit-wie-die-tabelle-t15767.html
\makeatletter
\newcommand\LastLTentrywidth{1em}
\newlength\longtablewidth
\setlength{\longtablewidth}{1in}
\newcommand{\getlongtablewidth}{\begingroup \ifcsname LT@\roman{LT@tables}\endcsname \global\longtablewidth=0pt \renewcommand{\LT@entry}[2]{\global\advance\longtablewidth by ##2\relax\gdef\LastLTentrywidth{##2}}\@nameuse{LT@\roman{LT@tables}} \fi \endgroup}

% \setlength{\parindent}{0.5in}
% \setlength{\parskip}{0pt plus 0pt minus 0pt}

% Overwrite redefinition of paragraph and subparagraph by the default LaTeX template
% See https://github.com/crsh/papaja/issues/292
\makeatletter
\renewcommand{\paragraph}{\@startsection{paragraph}{4}{\parindent}%
  {0\baselineskip \@plus 0.2ex \@minus 0.2ex}%
  {-1em}%
  {\normalfont\normalsize\bfseries\itshape\typesectitle}}

\renewcommand{\subparagraph}[1]{\@startsection{subparagraph}{5}{1em}%
  {0\baselineskip \@plus 0.2ex \@minus 0.2ex}%
  {-\z@\relax}%
  {\normalfont\normalsize\itshape\hspace{\parindent}{#1}\textit{\addperi}}{\relax}}
\makeatother

\makeatletter
\usepackage{etoolbox}
\patchcmd{\maketitle}
  {\section{\normalfont\normalsize\abstractname}}
  {\section*{\normalfont\normalsize\abstractname}}
  {}{\typeout{Failed to patch abstract.}}
\patchcmd{\maketitle}
  {\section{\protect\normalfont{\@title}}}
  {\section*{\protect\normalfont{\@title}}}
  {}{\typeout{Failed to patch title.}}
\makeatother

\usepackage{xpatch}
\makeatletter
\xapptocmd\appendix
  {\xapptocmd\section
    {\addcontentsline{toc}{section}{\appendixname\ifoneappendix\else~\theappendix\fi\\: #1}}
    {}{\InnerPatchFailed}%
  }
{}{\PatchFailed}
\keywords{media-multitasking, cognitive flexibility, polynomial regression, MMM-S, Modified Card Sorting Test}
\DeclareDelayedFloatFlavor{ThreePartTable}{table}
\DeclareDelayedFloatFlavor{lltable}{table}
\DeclareDelayedFloatFlavor*{longtable}{table}
\makeatletter
\renewcommand{\efloat@iwrite}[1]{\immediate\expandafter\protected@write\csname efloat@post#1\endcsname{}}
\makeatother
\usepackage{csquotes}
\makeatletter
\renewcommand{\paragraph}{\@startsection{paragraph}{4}{\parindent}%
  {0\baselineskip \@plus 0.2ex \@minus 0.2ex}%
  {-1em}%
  {\normalfont\normalsize\bfseries\typesectitle}}

\renewcommand{\subparagraph}[1]{\@startsection{subparagraph}{5}{1em}%
  {0\baselineskip \@plus 0.2ex \@minus 0.2ex}%
  {-\z@\relax}%
  {\normalfont\normalsize\bfseries\itshape\hspace{\parindent}{#1}\textit{\addperi}}{\relax}}
\makeatother

\ifLuaTeX
  \usepackage{selnolig}  % disable illegal ligatures
\fi
\usepackage{bookmark}
\IfFileExists{xurl.sty}{\usepackage{xurl}}{} % add URL line breaks if available
\urlstyle{same}
\hypersetup{
  pdftitle={Media Multitasking and Cognitive Flexibility: An Investigation of a Non-linear Correlation},
  pdfauthor={Manuel Althaler 12032737},
  pdflang={en-EN},
  pdfkeywords={media-multitasking, cognitive flexibility, polynomial regression, MMM-S, Modified Card Sorting Test},
  hidelinks,
  pdfcreator={LaTeX via pandoc}}

\title{Media Multitasking and Cognitive Flexibility: An Investigation of a Non-linear Correlation}
\author{Manuel Althaler 12032737\textsuperscript{}}
\date{}


\shorttitle{Media Multitasking and Cognitive Flexibility}

\affiliation{\vspace{0.5cm}\textsuperscript{} Faculty of Psychology, University of Vienna, 200233 SE Bachelorarbeit (2024S), Bence Szaszkó BSc MA MSc}

\abstract{%
Research on media multitasking is troubled by ambiguous results, often comparing extreme groups of media multitasking behavior. This study investigated a potential non-linear correlation by using all data from the short media multitasking measure and the Modified Card Sorting Test to gain a comprehensive understanding of the relationship. Anticipating an inverse U-shaped correlation, the study employed a novel approach by using polynomial regression. A total of 161 participants were tested online via survey exchange sites. By comprehensively exploring the impact of media multitasking on cognitive flexibility, this study aimed to address various conflicting aspects of the current literature. However, no significant relationships were found, and underlying limitations were discussed to guide future research in this area.
}



\begin{document}
\maketitle

With the increased use of media technology and media multitasking (MM), there is a growing interest in the influence on our cognitions and behaviors (Carrier et al., 2009). One of the first studies that researched MM as a trait was conducted by Ophir et al. (2009). In their paper, MM was defined as the simultaneous consumption of different streams of content through different forms of media. The present work examines media multitasking as defined by the Media Multitasking Index, a questionnaire devised by Ophir and colleagues in the course of their work. This questionnaire categorizes individuals into Heavy (HMM), Light (LMM), and Intermediate Media Multitasking Users (IMM) based on standard deviations from the mean of the current sample. Future studies also used oter cut-off methods based on either quantiles or percentiles (Van Der Schuur et al., 2015). Further explanations of the MMI will be provided in the methods section. A review of the effects of media multitasking on youth by Van Der Schuur et al. (2015) found that MM was primarily investigated regarding three different aspects: cognitive control abilities, academic performance, and socioemotional function. In their review, the authors formulated two opposing hypotheses regarding the effects of MM on cognitive control: the scattered attention hypothesis and the trained attention hypothesis. The arguments supporting each one can be traced back to the discussion presented by Ophir et al. (2009). According to the scattered attention hypothesis, regular media multitasking leads to not only a ``breadth-bias'' toward media consumption but also a breadth-bias in cognitive control, which makes them susceptible to distractors. Conversely, the trained attention hypothesis argues that the ability to switch between tasks and focus on relevant stimuli can be developed through training. Since then, there has been evidence for either the scattered attention hypotheses (Kong et al., 2023; Ophir et al., 2009; Uncapher \& Wagner, 2018; Van Der Schuur et al., 2015; Yap \& Lim, 2013) or the trained attention hypothesis (Alzahabi \& Becker, 2013; Ophir et al., 2009; Van Der Schuur et al., 2015). However, not all studies have found significant relationships; for instance, Edwards and Shin (2017) and Seddon et al. (2018) reported no significant effects of media multitasking on cognitive control.

\subsection{Media Multitasking and Cognitive Control}\label{media-multitasking-and-cognitive-control}

\begin{verbatim}
Cognitive control is separated into three different aspects: cognitive flexibility, working memory, and inhibitory control [@davidsonDevelopmentCognitiveControl2006]. Working memory is the ability to temporarily hold and manipulate information necessary for cognitive tasks, allowing individuals to process and use relevant information [@baddeleyDevelopmentsConceptWorking1994]. Cognitive flexibility is the competence to adapt and switch between different cognitive processes or tasks. Inhibitory control is the ability to suppress or override automatic responses, impulses, or distractions, allowing individuals to focus on relevant information and make intentional, goal-directed decisions [@diamondExecutiveFunctions2013]. Every aspect has its unique relationship with MM [@uncapherMindsBrainsMedia2018]. A recent meta-analysis by @kongCognitiveControlAdolescents2023, which examined the effect of MM on cognitive control while considering the different subsets as moderators, found a significant negative impact. The authors also found a significant moderating effect of type for working memory and inhibitory control, while being non-significant for cognitive flexibility. Other reviews, however, did not come to the same decisive conclusions [@kobayashiRelationshipMediaMultitasking2020; @uncapherMindsBrainsMedia2018]. While the research regarding cognitive flexibility argues for a non-significant relationship, there are considerations that these findings may result from comparing extreme groups. For example, some research found that IMMs performed better than HMMs on tests of focused attention, suggesting a possible inverse U-shaped correlation between MM and components of cognitive control [@cardoso-leiteTechnologyConsumptionCognitive2016; @shinModerateAmountsMedia2020]. @shinModerateAmountsMedia2020 demonstrated that IMMs outperformed both HMMs and LLMs on a more challenging variant of the n-back task, as opposed to the easier versions. Additionally, HMMs did not score significantly differently from LMMs, which aligns with the inverse U-shape hypothesis. To better understand the subtle differences influencing the effects of MM on cognitive flexibility, it is important to examine the full spectrum of MM behavior. Thus, this study explores the shape of the relationship between media multitasking and cognitive flexibility without excluding data based on MM index scores. This study postulates a non-linear correlation between MM and cognitive flexibility, and further, it postulates that the non-linear relationship follows an inverse u-shape.
\end{verbatim}

\section{Methods}\label{methods}

We report how we determined our sample size, all data exclusions (if any), all manipulations, and all measures in the study.

\subsection{Participants}\label{participants}

\subsection{Material}\label{material}

\subsection{Procedure}\label{procedure}

\subsection{Data analysis}\label{data-analysis}

We used R (Version 4.4.1; R Core Team, 2024) and the R-packages \emph{papaja} (Version 0.1.2.9000; Aust \& Barth, 2023), and \emph{tinylabels} (Version 0.2.4; Barth, 2023) for all our analyses.

\section{Results}\label{results}

\section{Discussion}\label{discussion}

\newpage

\section{References}\label{references}

\phantomsection\label{refs}
\begin{CSLReferences}{1}{0}
\bibitem[\citeproctext]{ref-alzahabiAssociationMediaMultitasking2013}
Alzahabi, R., \& Becker, M. W. (2013). The association between media multitasking, task-switching, and dual-task performance. \emph{Journal of Experimental Psychology: Human Perception and Performance}, \emph{39}(5), 1485--1495. \url{https://doi.org/10.1037/a0031208}

\bibitem[\citeproctext]{ref-R-papaja}
Aust, F., \& Barth, M. (2023). \emph{{papaja}: {Prepare} reproducible {APA} journal articles with {R Markdown}} {[}Manual{]}.

\bibitem[\citeproctext]{ref-R-tinylabels}
Barth, M. (2023). \emph{{tinylabels}: {Lightweight} variable labels} {[}Manual{]}.

\bibitem[\citeproctext]{ref-carrierMultitaskingGenerationsMultitasking2009}
Carrier, L. M., Cheever, N. A., Rosen, L. D., Benitez, S., \& Chang, J. (2009). Multitasking across generations: {Multitasking} choices and difficulty ratings in three generations of {Americans}. \emph{Computers in Human Behavior}, \emph{25}(2), 483--489. \url{https://doi.org/10.1016/j.chb.2008.10.012}

\bibitem[\citeproctext]{ref-edwardsMediaMultitaskingImplicit2017}
Edwards, K. S., \& Shin, M. (2017). Media multitasking and implicit learning. \emph{Attention, Perception, \& Psychophysics}, \emph{79}(5), 1535--1549. \url{https://doi.org/10.3758/s13414-017-1319-4}

\bibitem[\citeproctext]{ref-kongCognitiveControlAdolescents2023}
Kong, F., Meng, S., Deng, H., Wang, M., \& Sun, X. (2023). Cognitive {Control} in {Adolescents} and {Young Adults} with {Media Multitasking Experience}: A {Three-Level Meta-analysis}. \emph{Educational Psychology Review}, \emph{35}(1), 22. \url{https://doi.org/10.1007/s10648-023-09746-0}

\bibitem[\citeproctext]{ref-ophirCognitiveControlMedia2009}
Ophir, E., Nass, C., \& Wagner, A. D. (2009). Cognitive control in media multitaskers. \emph{Proceedings of the National Academy of Sciences}, \emph{106}(37), 15583--15587. \url{https://doi.org/10.1073/pnas.0903620106}

\bibitem[\citeproctext]{ref-R-base}
R Core Team. (2024). \emph{R: A language and environment for statistical computing} {[}Manual{]}. R Foundation for Statistical Computing.

\bibitem[\citeproctext]{ref-seddonExploringRelationshipExecutive2018}
Seddon, A. L., Law, A. S., Adams, A.-M., \& Simmons, F. R. (2018). Exploring the relationship between executive functions and self-reported media-multitasking in young adults. \emph{Journal of Cognitive Psychology}, \emph{30}(7), 728--742. \url{https://doi.org/10.1080/20445911.2018.1525387}

\bibitem[\citeproctext]{ref-uncapherMindsBrainsMedia2018}
Uncapher, M. R., \& Wagner, A. D. (2018). Minds and brains of media multitaskers: {Current} findings and future directions. \emph{Proceedings of the National Academy of Sciences}, \emph{115}(40), 9889--9896. \url{https://doi.org/10.1073/pnas.1611612115}

\bibitem[\citeproctext]{ref-vanderschuurConsequencesMediaMultitasking2015}
Van Der Schuur, W. A., Baumgartner, S. E., Sumter, S. R., \& Valkenburg, P. M. (2015). The consequences of media multitasking for youth: {A} review. \emph{Computers in Human Behavior}, \emph{53}, 204--215. \url{https://doi.org/10.1016/j.chb.2015.06.035}

\bibitem[\citeproctext]{ref-yapMediaMultitaskingPredicts2013}
Yap, J. Y., \& Lim, S. W. H. (2013). Media multitasking predicts unitary versus splitting visual focal attention. \emph{Journal of Cognitive Psychology}, \emph{25}(7), 889--902. \url{https://doi.org/10.1080/20445911.2013.835315}

\end{CSLReferences}


\end{document}
